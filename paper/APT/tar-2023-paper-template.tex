% Paper template for TAR 2022
% (C) 2014 Jan Šnajder, Goran Glavaš, Domagoj Alagić, Mladen Karan
% TakeLab, FER

\documentclass[10pt, a4paper]{article}

\usepackage{tar2023}

\usepackage[utf8]{inputenc}
\usepackage[pdftex]{graphicx}
\usepackage{booktabs}
\usepackage{amsmath}
\usepackage{amssymb}

\title{Beyond Duplicates: Unleashing Transitivity in Quora Data Augmentation}

\name{Renato Jurišić, Mihael Miličević, Josip Srzić} 

\address{
University of Zagreb, Faculty of Electrical Engineering and Computing\\
Unska 3, 10000 Zagreb, Croatia\\ 
\texttt{\{renato.jurisic,mihael.milicevic,josip.srzic\}@fer.hr}\\
}
          
         
\abstract{ 
This document provides the instructions on formatting the TAR system description paper in \LaTeX{}. This is where you write the abstract (i.e., summary) of the work you carried out within the project. The abstract is a paragraph of text ranging between 70 and 150 words.
}

\begin{document}

\maketitleabstract

\section{Introduction}
- opisati duplicate question identification, zajedno s motivacijom
- opisati motivaciju iza augmentacije kao postupka
- opisati ukratko ideju vezanu za tranizitivnu augmentaciju
- opisati research questions - pomaže li pozitivna augmentacija, s kojim korakom, koliko treba dodavati negativnih primjera da bi ostalo izbalansirano (spoiler alert: ne triba)

\section{Related work}
- opisati QQP competition (kaggle)
- opisati ona dva rada što su došli uz zadatak
- opisati onaj Renatov rad - oni nisu istraživali korake i udio negativnih spram pozitivnih

\section{Augmentation with transitive relations}
- cilj: približiti se relaciji ekvivalencije
- zašto približiti a ne postići ju - zato što je dataset nepotpun i ako neki za neki par ne možemo augmentacijom doći do toga da su duplikati, ne prepostavljamo da nisu
- smisliti neki primjer gdje A i B jesu ekvivalentni, B i C jesu, a A i C ne moraju biti - malo je shaky, ali mislimo da na našem datasetu prolazi
- opisati vrlo precizno kako radimo augmentaciju i pozitivnih i negaitvnih primjera - dodati onaj graf
- opisati zašto dodavanjem pozitivnih primjera ne možemo dobiti negativne parove
- također opisati da nam je glavni cilj istražiti dodavanje pozitivnih parova, negativne dodajemo tek da vidimo trebalo li očuvati nebalansiranost dataseta
- pokažemo par primjera pozitivnih i negativnih generiranih parova, s različitim koracima

\section{Experimental setup}
- opisati dataset - QQP rad
- kako smo splittali podatke, kako smo pazili da nam ne leaka u test set blabla - brisali smo podatke koji su u test setu
- opisati model - citirati BERT rad
- optimizacija hiperparametara - opisati dobivene hiperparametre (radimo na datasetu BEZ augmentacije)
- opisati training setup
- eksperiment - ona tablica (citat: R.J.)
- referirati Pytorch i ŠegvićCloud

\section{Results}
- rezulati eksperimenta - pokazati negativne primjere koje ne može naučiti - naša interpretacija zašto
- tablice, statistički testovi, grafovi

\section{Conclusion}
- opisati što smo radili
- opisati rezultate koje smo dobili
- future work - istražiti druge modele, druge datasetove

\end{document}
